% Options here are passed to the article class.
% Most common options: 10pt, 11pt, 12pt
\documentclass[10pt]{datasheet}

% Input encoding and typographical rules for English language
\usepackage[utf8]{inputenc}
\usepackage[english]{babel}
\usepackage[english]{isodate}

% tikz is used to draw images in this example, but you can
% also use \includegraphics{}.
\usepackage{graphicx}

% These define global texts that are used in headers and titles.
\title{DC03: 10BPS 2 Digit Decimal Decoder}
\author{Andrews54757}
\tags{decoders, decimal, linear time}
\date{October 2022}
\revision{Revision 1}
\begin{document}
\maketitle

\section{Features}

\begin{itemize}
\item{O(n) linear time decoding, 2 gt/block. Signal travels 10 blocks per second.}
\item{Slim profile. 4 blocks wide.}
\item{Hopperspeed throughput.}
\item{QC based logic with state auto-fix line.}
\end{itemize}

\section{Applications}

\begin{itemize}
\item{Decoding decimal signals for use in encoded chest halls}
\end{itemize}

\section{General Description}
The DC03 decoder takes two decimal digits and outputs a pulse at one of 100 slices corresponding to the code. The signal travels 10 blocks per second giving the device an O(n) time complexity.
% Switch to next column
\vfill\break

\begin{figure}[h]
    \centering
    \includegraphics[width=0.48\textwidth]{bps.png}
    \caption{\centering 10 BPS 2 Digit Decimal Decoder}
\end{figure}

% For wide tables, a single column layout is better. It can be switched
% page-by-page.
\onecolumn

\section{Device Specifications}

\begin{table}[h]
    \caption{Inputs}
    \begin{tabularx}{\textwidth}{l | c | X}
        \thickhline
        \textbf{Name} & \textbf{Range} & \textbf{Description} \\
        \hline
        Digit 1 & 1-10 & First digit indicating horizontal section. \\
        Digit 2 & 1-10 & Second digit indicating slice position in section. \\
        \hline
        Execute & Pulse & Clock signal of device. \\
        Auto-fix & Pulse & Automatically fixes incorrect dropper states. \\
        \thickhline
\end{tabularx}
\end{table}

\begin{table}[h]
    \caption{Outputs}
    \begin{tabularx}{\textwidth}{l | c | X}
        \thickhline
        \textbf{Name} & \textbf{Range} & \textbf{Description} \\
        \hline
        Mapped signal & Pulse & Outputs to one of 100 slices corresponding to input code. \\
        \thickhline
\end{tabularx}
\end{table}

\begin{table}[h]
    \caption{Device Specifications}
    \begin{tabularx}{\textwidth}{l | c c c | c | X}
        \thickhline
        \textbf{Parameter} & \textbf{Min.} & \textbf{Typ.} & \textbf{Max.} &
        \textbf{Unit} & \textbf{Conditions} \\
        \hline
        Throughput  & 8 & - & - & gt & Normal Usage \\
        \hline
        Latency  & 26 & - & 224 & gt & Input to Output. \\
        \hline
        Active Lag & +0.2 & - & +1.6 & ms & At Hopperspeed. Ryzen 5 3600, 2GB RAM. MC 1.18.1 with Lithium. \\
        \hline
        MC Version & 1.13 & 1.18.2 & - & MCV & Latest version at time of writing: 1.19.2\\
        \hline
        Dimensions & & 104 x 8 x 4 & & Blocks & \\
        \thickhline
\end{tabularx}
\end{table}
\newpage
\section{Testing Data}
\begin{table}[h]
\caption{Executed Tests}
\begin{tabularx}{\textwidth}{l | X}
    \thickhline
    \textbf{Test} & \textbf{Result} \\
    \hline
    Code test & Device was able to decode all possible codes successfully.\\
    \hline
    Auto-fix test & Device was able to reset faulty dropper states successfully.\\
    \hline
    Throughput test & Device was able to decode at 8gt throughput.\\
    \thickhline
\end{tabularx}
\end{table}

\section{Download Information}
\begin{table}[h]
    \caption{Download Information}
    \begin{tabularx}{\textwidth}{l | l | l | X}
        \thickhline
        \textbf{Identifier} & \textbf{MC} & \textbf{File} & \textbf{Description} \\
        \hline
        DC03 & 1.18.2 & DC03\_10BPS\_decimal\_decoder\_1.18.2.litematic & Litematic of decoder. Includes subregions for testing. Does not include inventories. \\
        \hline
        \thickhline
    \end{tabularx}
\end{table}

\begin{figure}[h]
    \includegraphics[width=0.48\textwidth]{inv.png}
    \caption{Inventories Setup Guide}
\end{figure}

\end{document}

