% Options here are passed to the article class.
% Most common options: 10pt, 11pt, 12pt
\documentclass[10pt]{datasheet}

% Input encoding and typographical rules for English language
\usepackage[utf8]{inputenc}
\usepackage[english]{babel}
\usepackage[english]{isodate}

% tikz is used to draw images in this example, but you can
% also use \includegraphics{}.
\usepackage{graphicx}

% These define global texts that are used in headers and titles.
\title{LC03: Await Latch}
\author{Andrews54757}
\tags{logic-and-computation}
\date{December 2022}
\revision{Revision 1}
\begin{document}
\maketitle

\section{Features}

\begin{itemize}
\item{2 Wide and 2 High}
\item{"Unbreakable"}
\end{itemize}

\section{Applications}

\begin{itemize}
\item{Queueing tasks}
\end{itemize}

\section{General Description}
The LC03 Await Latch takes one pulse input and one boolean input. Pulse only goes through when boolean input is on. If it is off, it will wait until boolean input turns on and then send pulse. Won't break with random inputs.
\vfill\break

\begin{figure}[h]
    \centering
    \includegraphics[width=0.48\textwidth]{awaitlatch.png}
    \caption{\centering Await Latch}
\end{figure}

% For wide tables, a single column layout is better. It can be switched
% page-by-page.
\onecolumn

\section{Device Specifications}

\begin{table}[h]
    \caption{Inputs}
    \begin{tabularx}{\textwidth}{l | c | X}
        \thickhline
        \textbf{Name} & \textbf{Range} & \textbf{Description} \\
        \hline
        Pulse input & pulse & Pulsed signal to transport. \\
        \hline
        Ready input & 0-1 & Delays pulsed signal until ready. \\
        \thickhline
\end{tabularx}
\end{table}

\begin{table}[h]
    \caption{Outputs}
    \begin{tabularx}{\textwidth}{l | c | X}
        \thickhline
        \textbf{Name} & \textbf{Range} & \textbf{Description} \\
        \hline
        Output & Pulse & Output pulse \\
        \thickhline
\end{tabularx}
\end{table}

\begin{table}[h]
    \caption{Device Specifications}
    \begin{tabularx}{\textwidth}{l | c c c | c | X}
        \thickhline
        \textbf{Parameter} & \textbf{Min.} & \textbf{Typ.} & \textbf{Max.} &
        \textbf{Unit} & \textbf{Conditions} \\
        \hline
        Throughput  & 12 & - & - & gt & Normal Usage \\
        \hline
        Latency    & 4 & - & - & gt & From input to output \\
        \hline
        MC Version & 1.13 & 1.17.1 & - & MCV & Latest version at time of writing: 1.19.3\\
        \hline
        Dimensions & & 2 x 2 x 7 & & Blocks & \\
        \thickhline
\end{tabularx}
\end{table}
\newpage
\section{Testing Data}
\begin{table}[h]
\caption{Executed Tests}
\begin{tabularx}{\textwidth}{l | X}
    \thickhline
    \textbf{Test} & \textbf{Result} \\
    \hline
    Spam test & Device didn't break with spammed inputs. \\
    \hline
    Throughput test & Device was able to pulse signals successfully at 12gt throughput. \\
    \thickhline
\end{tabularx}
\end{table}

\section{Download Information}
\begin{table}[h]
    \caption{Download Information}
    \begin{tabularx}{\textwidth}{l | l | l | X}
        \thickhline
        \textbf{Identifier} & \textbf{MC} & \textbf{File} & \textbf{Description} \\
        \hline
        LC03 & 1.17.1 & \href{https://github.com/Soontech-Annals/Archive/blob/364bde8dbcbc2e5337489ff435bcda9b387017e2/Archive/logic-and-computation/LC03\%20Await\%20Latch/LC03\_await\_latch.litematic?raw=1}{LC03\_await\_latch.litematic} & Schematic of device. \\
        \hline
        \thickhline
    \end{tabularx}
\end{table}

\end{document}

