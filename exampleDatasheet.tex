% Options here are passed to the article class.
% Most common options: 10pt, 11pt, 12pt
\documentclass[10pt]{datasheet}

% Input encoding and typographical rules for English language
\usepackage[utf8]{inputenc}
\usepackage[english]{babel}
\usepackage[english]{isodate}

% tikz is used to draw images in this example, but you can
% also use \includegraphics{}.
\usepackage{graphicx}
\usepackage{float}
\usepackage{subcaption}

% These define global texts that are used in headers and titles.
\title{_Identifier: _Title}
\author{_Authors}
\tags{_CategoryTag, _OtherTags}
\date{_Date}
\revision{Revision _RevisionNumber}
\begin{document}
\maketitle

\section{Features}

\begin{itemize}
\item{_Feature}
\end{itemize}

\section{Applications}

\begin{itemize}
\item{_Application}
\end{itemize}

\section{General Description}
_Description

\vfill\break

\begin{figure}[H]
    \centering
    \includegraphics[width=0.48\textwidth]{_Image}
    \caption{\centering _ImageCaption}
\end{figure}

% For wide tables, a single column layout is better. It can be switched
% page-by-page.
\onecolumn

\section{Device Specifications}

\begin{table}[H]
    \caption{Inputs}
    \begin{tabularx}{\textwidth}{l | c | X}
        \thickhline
        \textbf{Name} & \textbf{Range} & \textbf{Description} \\
        \hline
        Clock signal & Pulse & Executes converion algorithm with each pulse. \\
        \thickhline
\end{tabularx}
\end{table}

\begin{table}[H]
    \caption{Outputs}
    \begin{tabularx}{\textwidth}{l | c | X}
        \thickhline
        \textbf{Name} & \textbf{Range} & \textbf{Description} \\
        \hline
        Output & Pulse & Outputs \\
        \thickhline
\end{tabularx}
\end{table}

\begin{table}[H]
    \caption{Device Specifications}
    \begin{tabularx}{\textwidth}{l | c c c | c | X}
        \thickhline
        \textbf{Parameter} & \textbf{Min.} & \textbf{Typ.} & \textbf{Max.} &
        \textbf{Unit} & \textbf{Conditions} \\
        \hline
        Throughput  & 8 & - & - & gt & Normal Usage \\
        \hline
        Latency    & 0 & - & - & gt & From input to dropper activation. \\
        \hline
        MC Version & 1.16 & 1.19.3 & - & MCV & Latest version at time of writing: 1.21.4\\
        \hline
        Dimensions & & 11 x 60 x 19 & & Blocks & \\
        \thickhline
\end{tabularx}
\end{table}
\section{Testing Data}
\begin{table}[H]
\caption{Executed Tests}
\begin{tabularx}{\textwidth}{l | X}
    \thickhline
    \textbf{Test} & \textbf{Result} \\
    \hline
    Throughput test & Device was able to function with 8gt clocked input. \\
    \thickhline
\end{tabularx}
\end{table}

\section{Download Information}
\begin{table}[H]
    \caption{Download Information}
    \begin{tabularx}{\textwidth}{l | l | l | X}
        \thickhline
        \textbf{Identifier} & \textbf{MC} & \textbf{File} & \textbf{Description} \\
        \hline
        _Identifier & 1.21.4 & _SchematicFile & Schematic of device. \\
        \hline
        \thickhline
    \end{tabularx}
\end{table}

\end{document}

